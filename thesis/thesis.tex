\documentclass[11pt,a4paper]{report}

\usepackage[utf8]{inputenc}
\usepackage{graphics}
\usepackage{amsmath}

\usepackage[hidelinks]{hyperref}
\usepackage{url} 

\usepackage{pdfpages}
\usepackage{pdflscape}
\usepackage{pythonhighlight}
\usepackage{amssymb}

\usepackage[resetlabels,labeled]{multibib}

\newcounter{counter}[section]
\renewcommand{\thecounter}{\thesection.\arabic{counter}}

\newenvironment{definition}[1]{\bigskip\refstepcounter{counter}\noindent\textbf{Definition \thecounter } \emph{#1} \par\nopagebreak\noindent \begin{itshape}}{\end{itshape}\bigskip}

\newenvironment{example}[1]{\bigskip\refstepcounter{counter}\noindent\textbf{Example \thecounter} \emph{#1} \par\nopagebreak}{\bigskip}

\newenvironment{notation}{\bigskip\refstepcounter{counter}\noindent\textbf{Notation \thecounter~}\nopagebreak \begin{itshape}}{\end{itshape}\bigskip}

\newenvironment{theorem}{\bigskip\refstepcounter{counter}\noindent\textbf{Theorem \thecounter }\nopagebreak \begin{itshape}}{\end{itshape}\medskip}

\newenvironment{lemma}{\bigskip\refstepcounter{counter}\noindent\textbf{Lemma \thecounter }\nopagebreak \begin{itshape}}{\end{itshape}\bigskip}

\newenvironment{remark}{\bigskip\refstepcounter{counter}\noindent\textbf{Remark \thecounter~ }\nopagebreak \begin{itshape}}{\end{itshape}\bigskip}

\begin{document}

%---------preamble generated from fithesis template
\includepdf[pages={2}]{preamble/pre.pdf}

\pagestyle{plain}
\pagenumbering{roman}

~\vfill

\section*{Acknowledgements}
I would like to thank ...


\newpage
%~\vfill
\section*{Abstract}
My abstract is as follow

\newpage
\section*{Keywords}
Keyword1, keyword2, keyword3...

\tableofcontents
\newpage

\setcounter{page}{0}
\pagenumbering{arabic}

%--------------------------------------------------------

\chapter{Introduction} \label{chap:intro}

Modelling complex systems in systems biology has to be conducted at several levels of abstraction that reflect well the known information~\cite{Kitano}. At every level, the system has to be described rigorously in a formal language that allows to avoid
misunderstanding and ambiguous interpretations. The more complex the system is, the harder it is to describe it rigorously while not losing human-readability and compactness of the description at the same time.

Traditional approaches used to describe biochemical systems are: a \emph{chemistry approach} employing ``mechanical'' descriptions by chemical reactions or a \emph{mathematical approach} using ordinary differential equations or other mathematical formalisms. An advantage of chemical reactions over mathematical equations is the fact that chemical reactions are composable to some extent, human-readable and well-understood across disciplines while still having an executable semantics. 
Moreover, there are methods automatising the generation of mathematical models from chemical equations.

The problem of both approaches is the \emph{scalability} which is understood at two
different levels: scalability of the model description (avoiding combinatorial explosion at the syntactic level) and scalability of the model execution (avoiding combinatorial explosion at the semantics level). Even when the formulation of a model does not run into syntactic scalability problems, the execution or simulation might be infeasible~\cite{Romers107136}. 

The so-called \emph{computational approach}~\cite{Cardelli,Henzinger} offers scalable model description by abstracting the information about individual model components and interactions. The scalability at the semantics level still remains to be a challenge. A promising computational approach is provided by \emph{rule-based modelling}~\cite{kappa_formal, BNGL} and process-algebraic frameworks~\cite{Cardelli,BioPEPA,BioSPI}. Rule-based models make a natural extension of the mechanical reaction-based models used in chemistry. Instead of operating with \emph{objects}, rule-based frameworks operate with \emph{types} that allow to avoid the combinatorial explosion that occurs when underlying objects are specified directly. The semantics of the model is given in terms of \emph{rules} defined on given types. An important advantage of rule-based approach is that mathematical models can be automatically generated from it. In particular, instead of relying on a single mathematical formalism, different mathematical models can thus be obtained for a given model (e.g., ODEs~\cite{KaDE}, PDEs~\cite{Smoldyn}, chemical master equation or continuous-time Markov chains~\cite{Pauleve2010,sneddon2011efficient}, reaction-diffusion systems~\cite{So2013}, etc.). 

The long-term aim of our research is the development of Comprehensive Modelling Platform~\cite{cs2bio2013,BCS} -- a general modelling framework that combines model building, model analysis, and annotation tasks in a single public site related to a single system. It respects the need for maintaining existing ODE models (which is still a typical scenario in systems biology) but allows to align them with a mechanistic rule-based description that is understandable by biologists, compact in size, and executable in terms of allowing basic analysis tasks ensuring consistency of the description. Such a comprehensive solution allows to support modellers effort in building mathematical models that have clear biochemical meaning and can be
easily integrated. Moreover, mechanistic descriptions can be later used as standalone computational models having all advantages of rule-based modelling. 

To that end, we have pioneered an idea of combining advantages of rule-based modelling with the simplicity of chemical reactions by introducing a rule-based language called \emph{Biochemical Space Language} (BCSL), introduced in~\cite{Ded201627}. It has the following key aspects: \emph{human-readability} -- easy to read, write, and maintain;
\emph{executability} -- formal executable semantics is defined allowing
efficient static analysis and consistency checking; \emph{universality} -- principally different cellular processes can be sufficiently combined in a single specification; and \emph{syntactic scalability} -- combinatorial explosion of the description is avoided. However, similarly as in any other rule-based language, the problem of the \emph{model execution scalability} is not covered efficiently.

We propose a scalable approach based on formal methods to analysis of rule-based models. The emphasis is placed on parameter synthesis using a symbolic method and static analysis enabled by unique level of abstraction enabled by rule-based approach. The methods will be demonstrated using BCSL but their applicability should be universal across other rule-based formalisms.

\chapter{Rule-based modelling}

First we define \emph{reaction-based system}.  

\begin{definition}{Multiset}
\emph{Multiset} $\mathtt{M}$ is a pair $(\mathcal{A}, \mathtt{m})$ where $\mathcal{A}$ is a set and $ \mathtt{m} : \mathcal{A} \rightarrow \mathbb{N} $ is a function from $\mathcal{A}$ to the set of natural numbers. The set $\mathcal{A}$ is called the \emph{reference set} of elements. For each element $\mathit{a}$ in $\mathcal{A}$ the \emph{multiplicity} (that is, number of occurrences) of $\mathit{a}$ is the number $\mathtt{M}(\mathit{a})$.
\end{definition}

The elements of the multiset are often referenced as \emph{species}. We denote with $\mathbb{M}$ the set of all multisets. With $\Gamma$ we denote the set of all arbitrary rational functions over the domain of objects. The notation $\varrho(\mathtt{M})$ with $\varrho \in \Gamma$ and $\mathtt{M} \in \mathbb{M}$ denotes the evaluation of function $\varrho$ to a rational number by substituting each variable $\mathit{a}$ by its corresponding value $\mathtt{M}(\mathit{a})$.

\begin{definition}{Reaction}
A reaction $\mathtt{react}$ is a triple $(\mathtt{lhs}, \mathtt{rhs}, \varrho) \in \mathbb{M} \times \mathbb{M} \times \Gamma$ where $\mathtt{lhs}$ is left-hand side, $\mathtt{rhs}$ is right-hand side, and $\varrho$ is a rate function.
\end{definition}

Informally, a reaction states how different species interact to change the number of objects of the corresponding species in the state of the system. The state of the system is an unstructured collection (multiset) of copies of each species. 

\begin{definition}{Reaction application}
Let $\mathtt{react} = (\mathtt{lhs}, \mathtt{rhs}, \varrho)$ be a reaction and $\mathtt{M}$ a multiset (state). The \emph{reaction application} $\mathtt{react}(\mathtt{M})$ is a pair $(\mathtt{M} - \mathtt{lhs} + \mathtt{rhs}, \varrho(\mathtt{M}))$ if $\mathtt{M} - \mathtt{lhs} > 0$, $(\mathtt{M}, 0)$ otherwise.
\end{definition}

Qualitatively, applying a reaction to a state (multiset) means removing from the state the elements that appear in the left-hand side of the reaction and adding the elements that appear on the right-hand side of the reaction. Additionally, the \emph{rate function} is evaluated in the process and provides reaction rate, i.e. at which the reaction happens. In the case when the number of occurrences of some species required by $\mathtt{lhs}$ is not high enough, the application is not possible.

\begin{definition}{Rule}
bla
\end{definition}

\begin{definition}{Rule application}
bla
\end{definition}

\begin{definition}{Rule-based system}
bla
\end{definition}

\begin{definition}{Rule-based system semantics}
bla
\end{definition}



$\mathtt{rule}$

\chapter{State of the Art} \label{chap:state}
 bla

 - add LBS

\section{Rule-based languages}
\label{rule_based_languages}

There are several rule-based languages dedicated for modelling of the biological systems. Each of them uses different features and abstractions. In this chapter, we will highlight the key features of some of the representatives.

\subsection{Kappa language}
\label{kappa}

The Kappa language~\cite{kappa_formal} was primary developed for the modelling of protein interactions. The key structures used in the language are \emph{agents} with \emph{binding sites}, which allow formation of \emph{bonds} between the agents. Each binding site must be unique with at most one bond. Each site can occur in one of several pre-defined states.

The Kappa \emph{rules} are changing properties of the agents. Particularly, the rule might add, delete, or change a bond or a state of one or multiple agents at once. The rules are patterns with two sides where each of them is a sequence of agents delimited by a comma. Same as all the other rule-based languages, when some context is not relevant, it is omitted. Example of a rule is in Figure~\ref{kappa-rule}.

\begin{figure}[!h]
\begin{center}
\begin{tabular}{c l}
(a) & A(bsa$\sim$\{ u, p \}) ; B(bsb$\sim$\{ a, i \}) \\
(b) & A(bsa), B(bsb$\sim$a) $\rightarrow$ A(bsa!1), B(bsb$\sim$a!1) \\
\end{tabular}
\end{center}
\caption{Example of a Kappa rule. (a) Definition of associated agents. There are allowed two agents, A and B, both with a binding site which might occur in two possible states. (b) The rule itself. It defines creation of bond on the sites of agents A and B such that site of agent B must be in an active state. Note the context with no importance for the interaction is omitted.}\label{kappa-rule}
\end{figure}

The general problem present in the Kappa language is too detailed description. For biological systems as defined in Chapter~\ref{Rule-based basics}, it might be often difficult to deal with all the bonds, especially in cases when we do not need such details.

\subsection{BioNetGen Language}
\label{bngl}

The BioNetGen Language (BNGL) \cite{BNGL} is very similar to the Kappa language and also has similar disadvantages. There are, however, a few general differences: 

\begin{enumerate}
	\item it is allowed to define multiple binding sites with same name for an agent,
	\item one binding site is allowed to have multiple bonds,
	\item in the rules, BNGL uses `+' and `.' for expressing reaction complex and complex of agents respectively.
\end{enumerate}

Example of a rule is given in Figure~\ref{bngl-rule}.

\begin{figure}[!h]
\begin{center}
\begin{tabular}{c l}
(a) & A(bsa$\sim$u$\sim$p) \\
  & B(bsb$\sim$a$\sim$i) \\
(b) & A(bsa) + B(bsb$\sim$a) $\rightarrow$ A(bsa!1).B(bsb$\sim$a!1) \\
\end{tabular}
\end{center}
\caption{Example of a BNGL rule. (a) Definition of associated agents. There are allowed two agents, A and B, both with a binding site which might occur in two possible states. (b) The rule itself. Note that the agents A and B must first create \emph{reaction complex} denoted by `+' (i.e., they must be close to each other) and then they create \emph{complex of agents} denoted by `.' (i.e., they are physically connected).}\label{bngl-rule}
\end{figure}

\subsection{Chromar}

The Chromar language~\cite{honorato2017chromar} allows to define attributes for agents and range them over pre-defined domains. The qualitative semantics are given by rule match on multisets composed of these agents producing a reaction. It is followed by standard application of the reaction (in the manner of multiset operations). The language is very useful when creating a model where we need to create new distinct objects and control population of these objects.

Embedding this language into the functional programming language Haskell increases its expressive power while making the ability of some analysis more expensive. Moreover, a user needs to understand Haskell (at least its basics) in order to use Chromar.

It is important to highlight a feature which the stochastic semantics of this language offers. Compared to the other rule-based languages, it is capable of specifying the rates for individual reactions inherited from the rule (Figure~\ref{chromar_rule}). It is allowed by variable value bindings and type-determination between left and right-hand side of the rule.

\begin{figure}[!h]
\begin{center}
A(a = x), A(a = y) $\xrightarrow[]{\text{f(x,y)}}$ A(a = x + y), A(a = y - 1) [g(x, y)]
\end{center}
\caption{Example of a Chromar rule. Arithmetic operations can be used in order to change properties of individual agents. Moreover, a rate function $f$ and conditional function $g$ increase applicability and practical use of the rule.}\label{chromar-rule}\label{chromar_rule}
\end{figure}

However, when it comes to readability and presentation to the user, the language is not the best choice. All the biologically relevant terms such as states have to be encoded in natural numbers.

\subsection{PySB}

The PySB language~\cite{Lopez646} works as a package in the Python programming language. The definition of the models directly in the code allows the usage of the full syntax of Python, what significantly increases the expression power of PySB. On the other hand, it might be hard to follow abstractions which are possible this way and models themselves are hard to analyse. Moreover, the core of the language is defined by translating to BNGL (Section~\ref{bngl}). An example of a rule is given in Figure~\ref{pysb_rule}.

\begin{figure}[!h]
\begin{center}
\begin{python}
# Declare the monomers
Monomer('L', ['s'])
Monomer('R', ['s'])

# Declare the binding rule
Rule(L(s=None) + R(s=None) <> L(s=1) % R(s=1), kf, kr)
\end{python}
\end{center}
\caption{Example of a PySB rule. The rule describes creation of a bond between agents $\mathtt{R}$ and $\mathtt{L}$.}\label{PySB-rule}\label{pysb_rule}
\end{figure}

\begin{figure}[!h]
\lstset{language=XML}
\begin{lstlisting}[basicstyle=\scriptsize, frame=single]
<?xml version="1.0" encoding="UTF-8"?>
<sbml xmlns="http://www.sbml.org/sbml/level2" level="2" version="1">
  <model>
    <listOfCompartments>
      <compartment id="c" constant="true" multi:isType="true" />
      <compartment id="cc" constant="true" multi:isType="true">
        <multi:listOfCompartmentReferences>
          <multi:compartmentReference multi:id="cr1" multi:compartment="c" />
          <multi:compartmentReference multi:id="cr2" multi:compartment="c" />
        </multi:listOfCompartmentReferences>
      </compartment>
    </listOfCompartments>
    <multi:listOfSpeciesTypes>
      <multi:bindingSiteSpeciesType multi:id="stA" multi:compartment="c" />
      <multi:speciesType multi:id="stAA" multi:compartment="cc">
        <multi:listOfSpeciesTypeInstances>
          <multi:speciesTypeInstance multi:id="stiA1" multi:speciesType="stA"
            multi:compartmentReference="cr1" />
          <multi:speciesTypeInstance multi:id="stiA2" multi:speciesType="stA"
            multi:compartmentReference="cr2" />
        </multi:listOfSpeciesTypeInstances>
        <multi:listOfInSpeciesTypeBonds>
          <multi:inSpeciesTypeBond multi:bindingSite1="stiA1" multi:bindingSite2="stiA2" />
        </multi:listOfInSpeciesTypeBonds>
      </multi:speciesType>
    </multi:listOfSpeciesTypes>
    <listOfSpecies>
      <species id="spA" multi:speciesType="stA" compartment="c" ... />
      <species id="spAA" multi:speciesType="stAA" compartment="cc" ... />
    </listOfSpecies>
    <listOfReactions>
      <reaction id="reaction" ...>
        <listOfReactants>
          <speciesReference id="r1" species="spA" multi:compartmentReference="cr1" ... />
          <speciesReference id="r2" species="spA" multi:compartmentReference="cr2" ... />
        </listOfReactants>
        <listOfProducts>
          <speciesReference species="spAA" ... />
        </listOfProducts>
        ...
      </reaction>
      ...
    </listOfReactions>
  </model>
</sbml>
\end{lstlisting}
\caption{Example of a simple SBML-multi model. The model requires specification of language level (version). In the definition, there are included SpeciesTypes, Species which belong to these types, and Reactions where those species are interacting. Additionally, all processes are encapsulated in the compartments.}\label{SBML_example}
\end{figure}

\subsection{SBML-multi package}

Systems Biology Markup Language (SBML)~\cite{hucka2003systems} is a standard established for systems biology. A SBML-multi package~\cite{zhang2015sbml} is able to describe all the necessary rule-based features and therefore it is possible to export each model in a rule-based language in this format. It is the most suitable format for exchange and storage of the models but less for analysis and direct representation to the users (it is an XML format). It serves as an intermediate language. For an example see Figure~\ref{SBML_example}.


\chapter{Aim of the Thesis} \label{chap:aim}

\section{Objectives and expected results}

The aim of the PhD thesis is to apply ...
The specific steps that are to be done in order to follow this aim are
following.
\begin{itemize}
\item 
\item 
\item 
\item
\end{itemize}

\section{Expected outputs}
Besides the TOPIC incorporating results of
all the above objectives, I would like to provide the following outputs.
\begin{itemize}
\item 
\item 
\item
\item 
\end{itemize}

\section{Progression schedule}
The plan of my future study and research activities, besides attending
courses and teaching, is following.
\begin{itemize}
\item Doctoral exam and defence of this thesis proposal -- January 20XY
\item Research on the  -- from now until January 20XY+2
\item A~version of the tool incorporating -- June 20XY
\item Implementing new methods in the tool -- June 20XY+1 till January 20XY+2
\item Final version of the tool -- January 20XY+2
\item Writing of the PhD thesis -- January 20XY+2 till June 20XY+2
\item Final version of the thesis -- June 20XY+2
\end{itemize}

% ------------------------------- bibliography

\newpage
\stepcounter{chapter}
\addcontentsline{toc}{chapter}{Bibliography}

\bibliographystyle{splncs03}
\bibliography{all}

% ------------------------------- appendix

\appendix

\chapter{Summary of Present Study Results}\label{app:results}

\section{Courses and Summer Schools}
I have attended the following courses.
\begin{itemize}
\item 
\item 
\item 
\item 
\end{itemize}

I have also participated in summer schools ...

\section{Research Results}

I have participated  SOME research project of the QW Laboratory supported by grants
GRANT NUMBER. As a~member of this team, I have ...

 My publications and presentations related
to the topic of my PhD thesis are listed below.

\subsection*{Publications}

\end{document}

